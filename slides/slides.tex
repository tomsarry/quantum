\documentclass{beamer}
\usepackage[utf8]{inputenc}
\usepackage[T1]{fontenc}
\usepackage{graphicx}
\usepackage{pifont}
\usepackage{amsmath,amsthm,amssymb,mathtools,enumerate}
\usepackage[francais]{babel}
\usepackage{datetime} 
\usepackage{nicefrac} 
\usepackage[normalem]{ulem}
\usepackage{tikz}
\usetikzlibrary{positioning,calc}
\usepackage{physics} 
\usepackage[short,nocomma]{optidef}


\usetheme{metropolis}
\usefonttheme[onlymath]{serif}
\setbeamertemplate{enumerate items}[default]
\setbeamertemplate{navigation symbols}{} %remove navigation symbols

\graphicspath{ {../images/} }

\DeclarePairedDelimiterX\set[1]\lbrace\rbrace{\def\given{\;\delimsize\vert\;}#1}

\title[Th\'eorie de la Complexit\'e Quantique]{Introduction \`a la Th\'eorie de Complexit\'e Quantique}
\author{Tom Sarry}
\institute[UdeM]{Universit\'e de Montr\'eal}
\date[]{\today}
\titlegraphic{\vspace{7cm}\flushright\includegraphics[height=1cm]{udem.png}} 

\newcommand{\bqp}{\textsf{\textbf{BQP}}}
\newcommand{\bpp}{\textsf{\textbf{BPP}}}

\begin{document}

\frame{\titlepage}

\begin{frame}
  \frametitle{Table des Mati\`eres}
  \tableofcontents
\end{frame}


\section{Introduction}

\begin{frame}
  \frametitle{Le Probl\`eme}

  \begin{enumerate}
    \item Machines de Turing (Programmes de Branchement, Circuits Bool\'eens,...) sont bas\'ees sur la \textbf{physique classique}
    \item On a de bonnes raisons de croire que le monde ob\'eit \`a la \textbf{physique quantique}
  \end{enumerate}
  Comment \'etudier la puissance du calcul quantique ?
\end{frame}

\begin{frame}
  \frametitle{Les Questions}

  \begin{enumerate}
    \item Utiliser les outils existants ? (Mod\`eles de Calcul, classes de complexit\'e)
    \item Situer la puissance de l'ordinateur quantique dans la hi\'erarchie connue
    \item R\'esoudre des questions ouvertes? \pause \textsf{P} $\neq$ \textsf{NP}
  \end{enumerate}
\end{frame}

\section{Mod\`eles de Calcul Quantique}

\begin{frame}
  \frametitle{Diff\'erents Mod\`eles}

  Extension logique: adapter les mod\`eles existants pour supporter le calcul quantique:
  \begin{itemize}
    \item TM (Turing Machine) $\rightarrow$ QTM (Quantum TM)
    \item Finite Automaton (FA) $\rightarrow$ QFA (Quantum FA)
    \item Boolean Circuits $\rightarrow$ Quantum Circuits: vu en classe!
    \item Branching Programs (BP) $\rightarrow$ QBP (Quantum BP)
  \end{itemize}
\end{frame}

\begin{frame}
  \frametitle{Dates Importantes}

  \begin{itemize}
    \item Turing 1936: Machines de Turing
    \item Cook 1971: Comment prouver $X \in \textsf{NP}$
    \item Baianu 1971: QFA
    \item Benioff 1980: Machine de calcul quantique
    \item Deutsch 1985: QTM
    \item Yao 1993: Circuit Quantique
    \item Bernstein, Verizani 1997: Universal QTM (avec co\^ut de simulation \emph{raisonnable})
    \item Nakanishi et al, 2000: Programme de Branchement Quantique
  \end{itemize}
\end{frame}


\section{\textbf{LA} classe de Compl\'exit\'e Quantique: \bqp}

\subsection{Circuits Quantiques}

\begin{frame}
  \frametitle{\bqp \ par Circuits Quantiques}

  $L \in \bqp$:
  \begin{itemize}
    \item $\iff \exists$ un algo. quantique qui r\'esolve le probl\`eme de d\'ecision avec haute probabilit\'e en temps poly.
    \item $\iff \exists$ une famille uniforme de temps poly de circuits quantiques $\set{Q_n:n\in \mathbb{N}}$ tel que:
          \begin{enumerate}
            \item $\forall n \in \mathbb{N}$, $Q_n$ prend $n$ qubits en entr\'ee et produit un bit en sortie
            \item $\forall x \in L,\ Pr(Q_{|x|}(x) = 1) \geq \nicefrac{2}{3}$
            \item $\forall x \notin L,\ Pr(Q_{|x|}(x) = 0) \geq \nicefrac{2}{3}$
          \end{enumerate}
  \end{itemize}
\end{frame}


\subsection{Machines de Turing Quantiques}

\begin{frame}
  \frametitle{Rappel: Machines de Turing}

  \begin{enumerate}
    \item Machines de Turing D\'eterministes: un chemin
    \item Machines de Turing Non D\'eterministes: plusieurs chemins
    \item Machines de Turing Probabilistes: plusieurs chemins, n'en suit qu'un
    \item Machines de Turing Quantique: \pause
          suit plusieurs chemins en superposition
  \end{enumerate}
\end{frame}

\begin{frame}
  \frametitle{Rappel: \bpp \ (Bounded-error Probabilistic Poly-time)}

  $L \in \bpp$: \ probl\`emes qui peuvent \^etre efficacement r\'esolus
  \begin{itemize}
    \item $\iff \exists$ un algo. qui r\'esolve le probl\`eme de d\'ecision avec haute probabilit\'e en temps poly.
    \item $\iff \exists$ une Machine de Turing Probabiliste $M$ tel que:
          \begin{enumerate}
            \item $M$ fonctionne en temps poly
            \item $\forall x \in L,\ Pr(M \downarrow x = 1) \geq \nicefrac{2}{3}$
            \item $\forall x \notin L,\ Pr(M \downarrow x = 0) \geq \nicefrac{2}{3}$
          \end{enumerate}
  \end{itemize}

  Peut diminuer la probabilit\'e d'erreur \`a $2^{-n^c}$ avec des votes majoritaires (Bornes de Chernoff)
\end{frame}

\begin{frame}
  \frametitle{\bqp \ (Bounded-error Quantum Poly-time) par QTM}

  $L \in \bqp$:
  \begin{itemize}
    \item $\iff \exists$ un algo. \alert{quantique} qui r\'esolve le probl\`eme de d\'ecision avec haute probabilit\'e en temps poly.
    \item $\iff \exists$ une Machine de Turing \alert{Quantique} $M$ tel que:
          \begin{enumerate}
            \item $M$ fonctionne en temps poly
            \item $\forall x \in L,\ Pr(M \downarrow x = 1) \geq \nicefrac{2}{3}$
            \item $\forall x \notin L,\ Pr(M \downarrow x = 0) \geq \nicefrac{2}{3}$
          \end{enumerate}
  \end{itemize}

  Peut diminuer la probabilit\'e d'erreur \`a $2^{-n^c}$ avec des votes majoritaires (Bornes de Chernoff)
\end{frame}

\begin{frame}
  \frametitle{Rappel: Machine de Turing Probabiliste}

  \textbf{PTM}: $M = (Q, \Sigma, \Gamma, \delta, q_0, F)$

  \begin{itemize}
    \item $Q$ \'etats
    \item $\Sigma$ alphabet d'entr\'ee
    \item $\Gamma$ alphabet bande de calcul ($\Sigma \subseteq \Gamma, \texttt{\#} \in \Gamma$)
    \item $\delta$ fonction de transition \\
          $\delta: Q \times \Gamma \times Q \times \Gamma \times \set{L, R} \rightarrow [0,1]$
    \item $q_0$ \'etat initial
    \item $F \subseteq Q$ \'etats acceptants
  \end{itemize}
\end{frame}

\begin{frame}
  \frametitle{Rappel: Machine de Turing Probabiliste}

  Pas tous les $\delta$ donnent des PTM valides!\\
  \pause
  $\sum_{q, \tau, d} \delta(p, \gamma, q, \tau, d) = 1, \quad \forall (p, \gamma) \in Q \times \Gamma$
\end{frame}

\begin{frame}
  \frametitle{Machine de Turing Quantique}

  \textbf{QTM} [BV97]: $M = (Q, \Sigma, \delta, q_0, F)$
  \begin{itemize}
    \item $Q$ \'etats
    \item $\Sigma$ alphabet
    \item $\delta$ machine de contr\^ole quantique $\delta: Q \times \Sigma \times \Sigma \times Q \times \set{L, R} \rightarrow \mathbb{C}$\\
          $\delta(p,\sigma,\tau,q,d)$: amplitude
    \item $q_0$ \'etat initial
    \item $F \subseteq Q$ \'etats acceptants
  \end{itemize}
\end{frame}

\begin{frame}
  \frametitle{Machine de Turing Quantique}

  \begin{itemize}
    \item Pas tous les $\delta$ donnent des QTM valides! \pause
    \item $\delta$ repr\'esente une \textbf{application lin\'eaire} $M_\delta$, nous d\'esirons qu'elle soit \alert{unitaire} \pause
    \item $M_\delta M_\delta^\dag = \mathbb{I} \iff$ pr\'eserve la norme $L_2$ ($M_\delta$ finie) \pause
    \item Nombre \alert{infini} de configurations (car bande de calcul) $\implies M_\delta$ infinie
  \end{itemize}
\end{frame}

\begin{frame}
  \frametitle{Machine de Turing Quantique}

  \textbf{Objectif:} $M_\delta M_\delta^\dag = \mathbb{I} \iff \delta$ valide

  \textbf{Theor\`eme (informel)} [BV97]: $\delta$ est valide ssi $M_\delta$ satisfait:
  \begin{enumerate}
    \item Colonnes de longueur unitaire
    \item Colonnes avec la m\^eme position de la t\^ete de lecture sont orthogonales
    \item Colonnes avec une position de t\^ete qui diff\`ere de 2 cases sont orthogonales
  \end{enumerate}
\end{frame}

\begin{frame}
  \frametitle{Machine de Turing Quantique}

  \textbf{Objectif:} Deux configurations $c_1,c_2$ qui pourraient donner lieu a la m\^eme configuration $c$ doivent \^etre orthogonales.
\end{frame}

\begin{frame}
  \frametitle{Machine de Turing Quantique}

  \textbf{Cas 1:} T\^etes de lecture sur la m\^eme case
  \begin{center}
    \begin{figure}
      \begin{tikzpicture}[every node/.style={block},
          block/.style={minimum height=1.5em,outer sep=0pt,draw,rectangle,node distance=0pt}]
        \node (A) {$\texttt{b}$};
        \node (B) [left=of A] {$\texttt{a}$};
        \node (C) [left=of B] {\ldots};
        \node (D) [right=of A] {$\texttt{c}$};
        \node (E) [right=of D] {\ldots};
        \node (F) [above = 0.75cm of A] {head};
        \draw[-latex] (F) -- (A);
        \draw (C.north west) -- ++(-1cm,0) (C.south west) -- ++ (-1cm,0)
        (E.north east) -- ++(1cm,0) (E.south east) -- ++ (1cm,0);
      \end{tikzpicture}
      \caption{$c_1$}
    \end{figure}
    \begin{figure}
      \begin{tikzpicture}[every node/.style={block},
          block/.style={minimum height=1.5em,outer sep=0pt,draw,rectangle,node distance=0pt}]
        \node (A) {$\texttt{z}$};
        \node (B) [left=of A] {$\texttt{a}$};
        \node (C) [left=of B] {\ldots};
        \node (D) [right=of A] {$\texttt{c}$};
        \node (E) [right=of D] {\ldots};
        \node (F) [above = 0.75cm of A] {head};
        \draw[-latex] (F) -- (A);
        \draw (C.north west) -- ++(-1cm,0) (C.south west) -- ++ (-1cm,0)
        (E.north east) -- ++(1cm,0) (E.south east) -- ++ (1cm,0);
      \end{tikzpicture}
      \caption{$c_2$}
    \end{figure}
  \end{center}
\end{frame}

\begin{frame}
  \frametitle{Machine de Turing Quantique}

  \textbf{Cas 2:} T\^etes de lecture diff\`erent de 2 cases
  \begin{center}
    \begin{figure}
      \begin{tikzpicture}[every node/.style={block},
          block/.style={minimum height=1.5em,outer sep=0pt,draw,rectangle,node distance=0pt}]
        \node (A) {$\texttt{b}$};
        \node (B) [left=of A] {$\texttt{z}$};
        \node (C) [left=of B] {\ldots};
        \node (D) [right=of A] {$\texttt{c}$};
        \node (E) [right=of D] {\ldots};
        \node (F) [above = 0.75cm of B] {head};
        \draw[-latex] (F) -- (B);
        \draw (C.north west) -- ++(-1cm,0) (C.south west) -- ++ (-1cm,0)
        (E.north east) -- ++(1cm,0) (E.south east) -- ++ (1cm,0);
      \end{tikzpicture}
      \caption{$c_1$}
    \end{figure}
    \begin{figure}
      \begin{tikzpicture}[every node/.style={block},
          block/.style={minimum height=1.5em,outer sep=0pt,draw,rectangle,node distance=0pt}]
        \node (A) {$\texttt{b}$};
        \node (B) [left=of A] {$\texttt{a}$};
        \node (C) [left=of B] {\ldots};
        \node (D) [right=of A] {$\texttt{c}$};
        \node (E) [right=of D] {\ldots};
        \node (F) [above = 0.75cm of D] {head};
        \draw[-latex] (F) -- (D);
        \draw (C.north west) -- ++(-1cm,0) (C.south west) -- ++ (-1cm,0)
        (E.north east) -- ++(1cm,0) (E.south east) -- ++ (1cm,0);
      \end{tikzpicture}
      \caption{$c_2$}
    \end{figure}
  \end{center}
\end{frame}

\begin{frame}
  \frametitle{Machine de Turing Quantique}

  \textbf{Cas 3:} T\^etes de lecture diff\`erent de 1 case ? \pause\\
  Non car \alert{obligation de bouger} dans la direction $d \in \set{L,R}$ !
  \begin{center}
    \begin{figure}
      \begin{tikzpicture}[every node/.style={block},
          block/.style={minimum height=1.5em,outer sep=0pt,draw,rectangle,node distance=0pt}]
        \node (A) {$\texttt{b}$};
        \node (B) [left=of A] {$\texttt{a}$};
        \node (C) [left=of B] {\ldots};
        \node (D) [right=of A] {$\texttt{c}$};
        \node (E) [right=of D] {\ldots};
        \node (F) [above = 0.75cm of A] {head};
        \draw[-latex] (F) -- (A);
        \draw (C.north west) -- ++(-1cm,0) (C.south west) -- ++ (-1cm,0)
        (E.north east) -- ++(1cm,0) (E.south east) -- ++ (1cm,0);
      \end{tikzpicture}
      \caption{$c_1$}
    \end{figure}
    \begin{figure}
      \begin{tikzpicture}[every node/.style={block},
          block/.style={minimum height=1.5em,outer sep=0pt,draw,rectangle,node distance=0pt}]
        \node (A) {$\texttt{b}$};
        \node (B) [left=of A] {$\texttt{a}$};
        \node (C) [left=of B] {\ldots};
        \node (D) [right=of A] {$\texttt{c}$};
        \node (E) [right=of D] {\ldots};
        \node (F) [above = 0.75cm of D] {head};
        \draw[-latex] (F) -- (D);
        \draw (C.north west) -- ++(-1cm,0) (C.south west) -- ++ (-1cm,0)
        (E.north east) -- ++(1cm,0) (E.south east) -- ++ (1cm,0);
      \end{tikzpicture}
      \caption{$c_2$}
    \end{figure}
  \end{center}
\end{frame}

\section{Analyse de \bqp}

\begin{frame}
  \frametitle{\bqp}

  \begin{itemize}
    \item \textbf{But:} \pause situer la classe, prouver inclusions
    \item \textbf{Comment:} \pause trouver un probl\`eme $X$ \bqp-complet \pause
    \item \textbf{Probl\`eme:} on n'en connait pas!
  \end{itemize}
\end{frame}

\begin{frame}
  \frametitle{\sout{\bqp} Promise-\bqp}

  \begin{itemize}
    \item entr\'ee $x$
    \item Promesse $Q(x)$
    \item Propri\'et\'e $R(x)$
  \end{itemize}
  \[\forall x[Q(x) \rightarrow [M \downarrow x = 1 \iff R(x)]]\]
\end{frame}

\begin{frame}
  \frametitle{Exemple}

  \begin{itemize}
    \item \textbf{Entr\'ee:} 1 qubit $\ket{x}$
    \item \textbf{Probl\`eme:} $\ket{x} = \ket{1}$ ? \pause
    \item \textbf{Promesse:} $\ket{x} = \ket{1} \lor \ket{x} = \frac{\sqrt{3}}{2} \ket{0} + \frac{1}{2} \ket{1}$
  \end{itemize}
\end{frame}

\begin{frame}
  \frametitle{\sout{\bqp} Promise-\bqp}

  \textbf{APPROX-QCIRCUIT-PROB (AQP):}
  \begin{itemize}
    \item \textbf{Entr\'ee:} Circuit Quantique $C$ sur $n$ qubits, $\alpha, \beta \in [0,1]$ \pause
    \item \textbf{Probl\`eme:} Mesure du premier qubit de $C\ket{0^n}$ donne $1$ avec probabilit\'e $\geq \alpha$? \pause
    \item \textbf{Promesse:} $\alpha > \beta$
  \end{itemize}

\end{frame}

\begin{frame}
  \frametitle{\sout{\bqp} Promise-\bqp}

  \textbf{AQP} est \bqp-hardu\\
  \textbf{Preuve (\'ebauche):}
  $\forall L \in \bqp$, utiliser \textbf{AQP} comme oracle avec $\alpha=\nicefrac{2}{3}, \beta=\nicefrac{1}{3}$
  \begin{enumerate}
    \item Fixer $x, Q_n$
    \item Construire $C_x$ tel que $C_x \ket{0^n}=\ket{x}$
    \item Joindre $C_x$ et $Q_n$ en un circuit $C'$
    \item Demander \`a l'oracle $(C', \nicefrac{2}{3}, \nicefrac{1}{3})$
  \end{enumerate}

\end{frame}

\begin{frame}
  \frametitle{R\'esultat 1: \bqp \ faible pour elle m\^eme}

  \[\bqp^{\bqp} = \bqp\]
\end{frame}

\begin{frame}
  \frametitle{R\'esultats 2: Inclusions (basiques)}

  \[\textsf{P} \subseteq \bpp \alert{\subseteq} \bqp \subseteq \textsf{PP} \subseteq \textsf{PSPACE} \subseteq \textsf{EXP} \]
\end{frame}

\begin{frame}
  \frametitle{Suppositions 1: Hi\'erarchie Polynomiale}

  \[\textsf{P} \subseteq \bpp {\subseteq} \bqp \subseteq \textsf{PP} \subseteq \textsf{PSPACE} \subseteq \textsf{EXP} \]
  \begin{itemize}
    \item \textbf{Rappel:} $\textsf{P} = \Sigma_0^p = \Pi_0^p \subseteq \textsf{NP} = \Sigma_1^p \subseteq \cup_k \Delta_k^p = \textsf{PH} \subseteq \textsf{PSPACE}$ \pause
    \item Fourier Sampling $\in \bqp$, $\notin \textsf{PH}?$
  \end{itemize}
\end{frame}

\section{Conclusion}

\begin{frame}
  \frametitle{Conclusion}

  \begin{itemize}
    \item Mod\`eles de Calcul Quantiques
    \item QTM
    \item \bqp \ (\textsf{QMA}, \textsf{EQP}, \textsf{AWPP}, ...)
  \end{itemize}
\end{frame}

\begin{frame}
  \frametitle{Sources}

  \begin{itemize}
    \item Bernstein, Verizani 1997, \emph{Quantum Complexity Theory}
    \item Deutsch 1985, \emph{Quatum Theory, the Church-Turing Principle and the Universal Quantum Computer}
    \item Fortnow 2003, \emph{One Complexity Theorist's View of Quantum Computing}
    \item \url{https://complexityzoo.net/}
  \end{itemize}
\end{frame}


\end{document}